\documentclass[preprint,12pt]{elsarticle}
% \documentclass[draft,12pt]{elsarticle}

\usepackage{hyperref}
\usepackage{graphicx}
\usepackage{subcaption}
\usepackage{amssymb}
\usepackage{amsmath}
\usepackage{multirow}
\usepackage{relsize}
\usepackage[utf8]{inputenc}
\usepackage{cleveref}
\usepackage{algorithm}
\usepackage[noend]{algpseudocode}
\usepackage[section]{placeins}
\usepackage{booktabs}
\usepackage{url}

% For the TODOs
\usepackage{xcolor}
\usepackage{xargs}
\usepackage[colorinlistoftodos,textsize=footnotesize]{todonotes}
\newcommand{\todoin}{\todo[inline]}
% from here: https://tex.stackexchange.com/questions/9796/how-to-add-todo-notes
\newcommandx{\unsure}[2][1=]{\todo[linecolor=red,backgroundcolor=red!25,bordercolor=red,#1]{#2}}
\newcommandx{\change}[2][1=]{\todo[linecolor=blue,backgroundcolor=blue!25,bordercolor=blue,#1]{#2}}
\newcommandx{\info}[2][1=]{\todo[linecolor=OliveGreen,backgroundcolor=OliveGreen!25,bordercolor=OliveGreen,#1]{#2}}

%Boldtype for greek symbols
\newcommand{\teng}[1]{\ensuremath{\boldsymbol{#1}}}
\newcommand{\ten}[1]{\ensuremath{\mathbf{#1}}}

\usepackage{lineno}
% \linenumbers

\journal{}

\begin{document}

\begin{frontmatter}

  \title{}
  \author[IITB]{Dinesh Adepu\corref{cor1}}
  \ead{adepu.dinesh.a@gmail.com}
  \author[University of Surrey]{Chuan Yu Wu}
  \ead{xyz@com}
\address[UoS]{Department of Aerospace Engineering, Indian Institute of
  Technology Bombay, Powai, Mumbai 400076}

\cortext[cor1]{Corresponding author}


\begin{abstract}

\end{abstract}

\begin{keyword}
%% keywords here, in the form: keyword \sep keyword
{Elastic solids collision}, {Frictional contact}, {SPH}, {Transport Velocity Formulation}

%% MSC codes here, in the form: \MSC code \sep code
%% or \MSC[2008] code \sep code (2000 is the default)

\end{keyword}

\end{frontmatter}

% \linenumbers

\section{Introduction}
\label{sec:intro}


\section{Numerical model}
\label{sec:numerical_model}
The SPH discretization of the continuity
equation~\cref{eq:sph-discretization-continuity} and the momentum equation
~\cref{eq:sph-momentum-fluid} respectively are,
\begin{equation}
  \label{eq:sph-discretization-continuity}
  \frac{d\overline{\rho}_i}{dt} = \sum_{j} \;
   m_j \; {\ten{u}}_{ij} \; \cdot \nabla_{i} W_{ij}(h),
\end{equation}

$\overline{\rho}_i=\epsilon_i \rho_i$.
\begin{equation}
  \label{eq:fluid-porosity}
  \epsilon_i = 1 - \sum_b W_{ib}(h_c) V_b
\end{equation}

%
Similarly, the discretized momentum equation for fluids is written as,
\begin{multline}
  \label{eq:sph-momentum-fluid}
  \frac{d\ten{u}_{i}}{dt} = - \sum_{j} m_j
  \bigg(\frac{p_i}{\overline{\rho}_i^2} + \frac{p_j}{\overline{\rho}_j^2}\bigg)
  \nabla_{i} W_{ij}
 \;+\;
  \sum_{j} m_j \frac{4 \eta \nabla W_{ij}\cdot
    \ten{r}_{ij}}{(\overline{\rho}_i + \overline{\rho}_j) (r_{ij}^2 + 0.01 h_{ij}^2)} \ten{u}_{ij}  \;+\;
  \ten{g}_{i},
\end{multline}
where $\ten{I}$ is the identity matrix, $\eta$ is the kinematic viscosity of the
fluid and \cite{morris1997modeling} formulation is used to discretize the
viscosity term.

We add to the momentum equation an additional artificial viscosity term
$\Pi_{ij}$~\cite{monaghan-review:2005} to maintain the stability of the
numerical scheme, given as,
\begin{align}
  \label{eq:mom-av}
  \Pi_{ij} =
  \begin{cases}
\frac{-\alpha h_{ij} \bar{c}_{ij} \phi_{ij}}{\bar{\rho}_{ij}}
  & \ten{u}_{ij}\cdot \ten{r}_{ij} < 0, \\
  0 & \ten{u}_{ij}\cdot \ten{r}_{ij} \ge 0,
\end{cases}
\end{align}
where,
%
\begin{equation}
  \label{eq:av-phiij}
  \phi_{ij} = \frac{\ten{u}_{ij} \cdot \ten{r}_{ij}}{r^2_{ij} + 0.01 h^2_{ij}},
\end{equation}
%
where $\ten{r}_{ij} = \ten{r}_i - \ten{r}_j$,
$\ten{u}_{ij} = \ten{u}_i - \ten{u}_j$, $h_{ij} = (h_i + h_j)/2$,
$\bar{\overline{\rho}}_{ij} = (\overline{\rho}_i + \overline{\rho}_j)/2$, $\bar{c}_{ij} = (c_i + c_j) / 2$, and
$\alpha$ is the artificial viscosity parameter.  The pressure $p_i$ is evaluated
using an equation of state:
\begin{equation}
\label{eqn:sph-eos}
  p_i = K \bigg(\frac{\overline{\rho}_i}{\epsilon \; \rho_{0}} - 1 \bigg).
\end{equation}
Where, $K=\rho_0 \, c_0^2$ is bulk modulus of the body, with
$c_0=10 \times V_{\text{max}}$ is speed of sound, while $\rho_0$ as the
initial density of the particles.



\FloatBarrier%
\section{Coupling forces on solid phase}
\label{sec:solid-fluid-coupling-force}
% Taken from Dike failure paper \cite{zhang_numerical_2023}

\begin{equation}
\label{eqn:sph-eos}
  \ten{f}_i = \ten{f}_i^{\text{buoyancy}} + \ten{f}_i^{\text{drag}}
\end{equation}

\begin{equation}
\label{eqn:f-buoyancy}
  \ten{f}_i^{\text{buoyancy}} = - V_i \sum_{a}
  \frac{m_a p_a}{\overline{\rho}_a}
  \nabla_{i} W_{ia} (h_c)
\end{equation}


\begin{equation}
\label{eqn:f-drag}
\ten{f}_i^{\text{drag}} = \frac{\beta_i V_i}{1 - \epsilon_i} (\ten{v}_i - \ten{u}_i)
\end{equation}

\begin{equation}
\label{eqn:epsilon_solid}
\epsilon_i = \frac{1}{\sum_a \frac{m_a}{\rho_a} W_{ia} (h_c)} \sum_a \epsilon_a \frac{m_a}{\rho_a} W_{ia} (h_c)
\end{equation}

\begin{equation}
\label{eqn:epsilon_solid}
\ten{v} = \frac{1}{\sum_a \frac{m_a}{\rho_a} W_{ia} (h_c)} \sum_a \ten{u}_a \frac{m_a}{\rho_a} W_{ia} (h_c)
\end{equation}

The inter-phase momentum exchange coefficient $\beta_i$ is expressed as:

\begin{align}
  \label{eq:analytical-x-cm-rolling-cylinder}
  \beta_{i} =
  \begin{cases}
    150 \frac{(1 - \epsilon_i)^2}{\epsilon_i}\frac{\mu_f}{d_i^2} + 1.75 (1 - \epsilon_i)\frac{\rho_f}{d_i} | \ten{v}_i - \ten{u}_i|  & \epsilon_i \leq 0.8, \\
    0.75 C_d \frac{\epsilon_i(1 - \epsilon_i)}{d_i} \rho_f | \ten{v}_i - \ten{u}_i| \epsilon_i^{-2.65} & \epsilon_i > 0.8 \\
\end{cases}
\end{align}
where $\mu_f, d_i, \rho_f$ and $C_d$ are the dynamic viscosity of the fluid,
diameter of particle $i$, density of the fluid and drag coefficient
respectively. The drag coefficient $C_d$ is given by:
\begin{align}
  \label{eq:analytical-x-cm-rolling-cylinder}
  C_d =
  \begin{cases}
    \frac{24}{Re_i}(1 + 0.15 Re_i^{0.687}) & Re_i \leq 1000, \\
    0.44 & Re_i > 1000 \\
\end{cases}
\end{align}
where $Re_i$ is the particle Reynolds number of particle 1:

\begin{equation}
Re_i = \frac{\rho_f d_i \epsilon_i | \ten{v}_i - \ten{u}_i|}{\mu_f}
\end{equation}



\FloatBarrier%
\section{Forces on the fluid due to interaction with solid}
\label{sec:coupling-force-on-fluid}


\begin{equation}
\ten{f}_a = -\frac{m_a}{\rho_a} \sum_i \frac{1}{S_i} \ten{f}_i W_{ai}(h_c)
\end{equation}
\begin{equation}
\ten{S}_i = \sum_a \frac{m_a}{\rho_a} W_{ia}(h_c)
\end{equation}

\FloatBarrier%
\section{Conclusions}
\label{sec:conclusions}


\section*{References}


\bibliographystyle{model6-num-names}
\bibliography{references}
\todoin{Virtual Experiments of Particle Mixing Process with the SPH-DEM Model}
add this citation
\end{document}

% ============================
% Table template for reference
% ============================
% \begin{table}[!ht]
%   \centering
%   \begin{tabular}[!ht]{ll}
%     \toprule
%     Quantity & Values\\
%     \midrule
%     $L$, length of the domain & 1 m \\
%     Time of simulation & 2.5 s \\
%     $c_s$ & 10 m/s \\
%     $\rho_0$, reference density & 1 kg/m\textsuperscript{3} \\
%     Reynolds number & 200 \& 1000 \\
%     Resolution, $L/\Delta x_{\max} : L/\Delta x_{\min}$ & $[100:200]$ \& $[150:300]$\\
%     Smoothing length factor, $h/\Delta x$ & 1.0\\
%     \bottomrule
%   \end{tabular}
%   \caption{Parameters used for the Taylor-Green vortex problem.}%
%   \label{tab:tgv-params}
% \end{table}

%%% Local Variables:
%%% mode: latex
%%% TeX-master: "paper"
%%% fill-column: 78
%%% End:
